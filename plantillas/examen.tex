% !TEX program = pdflatex
% !TEX encoding = UTF-8 Unicode

% Plantilla examen UGR
% Francisco Torralbo Torralbo, ftorralbo@ugr.es


% -------------------------------------------------------------------
% INFORMACIÓN DEL DOCUMENTO
% -------------------------------------------------------------------

% Datos del examen
\Grado{Grado}
\Departamento{Departamento}
\Asignatura{Asignatura}
\CursoAcademico{Curso académico 23/24}
\Convocatoria{ordinario}
\Fecha{17 de julio de 2024}
\Instrucciones{Todas las respuestas han de estar razonadas. Todos los ejercicios tienen la misma puntuación. Las hojas deben entregarse numeradas y llevar todas el nombre y grupo, situando esta hoja la primera.}


% Metadatos del archivo pdf generado
\hypersetup{%
  pdfsubject={Examen \printConvocatoria, \printGrado, \textcopyright Universidad de Granada},%
	pdfkeywords={examen, Universidad de Granada},%
}

\begin{document}

\maketitle

% Texto del examen
\begin{enumerate}
  \item Primer ejercicio
  \item \textbf{[2 puntos]} Segundo ejercicio
    \begin{enumerate}
      \item Primera cuestión
      \item Segunda cuestión
    \end{enumerate}
\end{enumerate}


\vfill

{
  \footnotesize
  \printInstrucciones
}


% Si es necesario que el documento tenga varias páginas descomentar el siguiente bloque:

% \newpage
% \thispagestyle{empty}
%
% Texto en nueva página.

\end{document}
